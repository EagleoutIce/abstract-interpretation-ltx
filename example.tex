\documentclass{article}

\errorcontextlines=9999

\usepackage[T1]{fontenc}
\usepackage[utf8]{inputenc}

\usepackage{absint}
\usepackage[prefix=]{xcolor-material}
\usepackage[colorlinks=true,allcolors=purple]{hyperref}

\usepackage{newtxmath,newtxtext}
\usepackage{enumitem}
\usepackage{microtype}
\usepackage{parskip}

\begin{document}
Hey! This is: \absintmark{lor}\absexpr{\lor}. \absintmark{iff}\absexpr{\iff}.


\absexpr{a \lor b \asdef\iff c \eq 3 \asdef\eq 4}\\
\(a \lor b \iff c\)

Lattice: A lattice \absexpr{(X, \sqsubseteq, \sqcup, \sqcap)} is a poset (partially ordered set), such that \absexpr{\forall a, b \in X: a \sqcup b} and \absexpr{a \sqcap b} exist (Mine).

Complete Lattice: A complete lattice \absexpr{(X, \sqsubseteq, \sqcup, \sqcap, \bot, \top)} is a lattice, such that
\begin{enumerate}[nosep]
   \item \absexpr{\forall S \subseteq X: \bigsqcup S} and \absexpr{\bigsqcap S} exist.
   \item \absexpr{X} has a least element \absexpr{\bot} and a greatest element \absexpr{\top}.
\end{enumerate}

For any set \absexpr{X}, the powerset \absexpr{(\powerset(X), \subseteq, \cup, \cap, \emptyset, X)} is a complete lattice.
\end{document}